\documentclass[11pt]{article}
\usepackage[margin=1.5cm, top=1cm]{geometry}
\usepackage{amsmath,amssymb,scalerel}
\usepackage{bbm}
\usepackage{stmaryrd} % for symbols like \llbracket
\usepackage{mathabx}
\usepackage{bm}
\usepackage{xcolor}
\usepackage{siunitx}
\usepackage{mathtools}
\usepackage{mathrsfs}
\usepackage[scr=rsfs,cal=boondox]{mathalfa}

\usepackage{array}
\usepackage{graphicx}

\usepackage[style=apa,backend=biber]{biblatex}
\addbibresource{../ref.bib}


\newcolumntype{A}{c@{\hspace{.5cm}}}
\newcolumntype{B}{c@{\hspace{2cm}}}
\newcolumntype{L}{>$l<$} 
\setlength{\jot}{6pt}
\renewcommand{\arraystretch}{1.5}

\title{2 phase flow formulation - assessing the mesh dependency}
\author{Renato Poli , Bruno Fernandes}
\begin{document}
\maketitle
\hrule

% some accelarators
\def\qq{\bm{q}}
\def\vv{\mathbf{v}}
\def\uu{\mathbf{u}}

\section{Problem statement}

We start from the continuity equation for each phase, all written in massic form:
\[
  0 = \frac{mass}{time} + \text{net outflow} - \text{sink}   
\]
that leads to
\[
  0 = \dot{\zeta_\alpha} + \nabla\cdot\uu_\alpha - q_\alpha 
\]
where $\zeta_\alpha$ is the fluid mass per unit volume ($kg/m^3$), 
$\uu_\alpha$ is the massic flow ($kg/m^2/s$) and
$q_\alpha$ is the sink term ($kg/m^3/s$).
\begin{gather}
   \zeta_\alpha = \rho_\alpha \phi S_\alpha \\
   \uu_\alpha = \rho_\alpha \vv_\alpha 
\end{gather}
where $\vv$ is the darcy velocity.
\[
  \vv_\alpha = - \frac{1}{\mu_\alpha} \bm{K}_\alpha \cdot ( \nabla p_\alpha - \rho_\alpha \bm{g} ) 
\]
where $\rho_\alpha$ is the fluid density ($kg/m^3$), $\vv$ is the fluid velocity vector ($m/s$) and $q_m$ is the source term ($kg/m^3/s$).

Define densities:
\[
   \rho_\alpha \approx \rho^0_\alpha [ 1 + c_\alpha ( p_\alpha - p_\alpha^0 ) ]        \quad,
\]
fluid fraction in the reservoir (porosity):
\[
  \phi  \approx \phi_0 \left[ 1 + c_f ( p - p_0 ) \right]     \quad,
\]
and permeability tensor
\[
  \bm{K}_\alpha = \bm{K} k_{r\alpha}      \quad,
\]
where $\bm{K}$ is the absolute (intrinsic) permeability tensor and $k_{r\alpha}$ is the relative permeability of that phase.

Consider the saturation restrictions:
\[
  1 = \sum_\alpha S_\alpha      \quad,
\]
and cappilary pressure for a two-phase flow system
\[
  p_o = p_w + P_{cow}   \quad.
\]

The system of PDE is hence
{
\newcommand{\Gammaw}{\Gamma_{\text{w}}}
\newcommand{\Gammao}{\Gamma_{\text{o}}}
\newcommand{\Omegawell}{\Omega_{\text{well}}}
\newcommand{\zetaalpha}{\zeta_\alpha}
\[\begin{array}{ l l }
   0 = \sum_\alpha \Big\{ \dot{\zetaalpha} + \nabla\cdot\uu_\alpha - q_\alpha  \Big\} & \quad\quad \text{in } \Omega        \\ 
   0 = \dot{\zeta_w} + \nabla\cdot\uu_w - q_w                                           & \quad\quad \text{in } \Omega        \\ 
   1 = \sum_\alpha S_\alpha                                                             & \quad\quad \text{in } \Omega        \\
   0 = \nabla p_w                                                                       & \quad\quad \text{in } \Gammao      \\
   1 = S_o                                                                              & \quad\quad \text{in } \Omega \times t=0  \\
   q_\alpha = \hat{q}_\alpha                                                            & \quad\quad \text{in } \Omegawell  \quad \text{(wells interface in sink terms)}    \\
\end{array}\]
}

\newpage
\printbibliography
\end{document}
