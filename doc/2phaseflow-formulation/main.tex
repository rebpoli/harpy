\documentclass[11pt]{article}
\usepackage[margin=1.5cm, top=1cm]{geometry}
\usepackage{amsmath,amssymb,scalerel}
\usepackage{bbm}
\usepackage{stmaryrd} % for symbols like \llbracket
\usepackage{mathabx}
\usepackage{bm}
\usepackage{xcolor}
\usepackage{siunitx}
\usepackage{mathtools}
\usepackage{mathrsfs}
\usepackage[scr=rsfs,cal=boondox]{mathalfa}

\usepackage{array}
\usepackage{graphicx}

\usepackage[style=apa,backend=biber]{biblatex}
\addbibresource{../ref.bib}


\newcolumntype{A}{c@{\hspace{.5cm}}}
\newcolumntype{B}{c@{\hspace{2cm}}}
\newcolumntype{L}{>$l<$} 
\setlength{\jot}{6pt}
\renewcommand{\arraystretch}{1.5}

\title{2 phase flow formulation - assessing the mesh dependency}
\author{Renato Poli , Bruno Fernandes}
\begin{document}
\maketitle
\hrule

% some accelarators
\def\qq{\bm{q}}
\def\vv{\mathbf{v}}
\def\ww{\mathbf{w}}
\def\uu{\mathbf{u}}

\section{Problem statement}

We start from the continuity equation for each phase, all written in massic form:
\[
  0 = \frac{mass}{time} + \text{net outflow} - \text{sink}   
\]
that leads to
\[
  0 = \dot{\zeta_\alpha} + \nabla\cdot\uu_\alpha - q_\alpha 
\]
where $\zeta_\alpha$ is the fluid mass per unit volume ($kg/m^3$), 
$\uu_\alpha$ is the massic flow ($kg/m^2/s$) and
$q_\alpha$ is the sink term ($kg/m^3/s$).
\begin{gather}
   \zeta_\alpha = \rho_\alpha \phi S_\alpha \\
   \uu_\alpha = \rho_\alpha \vv_\alpha 
\end{gather}
where $\vv$ is the darcy velocity.
\[
  \vv_\alpha = - \frac{1}{\mu_\alpha} \bm{K}_\alpha \cdot ( \nabla p_\alpha - \rho_\alpha \bm{g} ) 
\]
where $\rho_\alpha$ is the fluid density ($kg/m^3$), $\vv$ is the fluid velocity vector ($m/s$) and $q_m$ is the source term ($kg/m^3/s$).

Define densities:
\[
   \rho_\alpha \approx \rho^0_\alpha [ 1 + c_\alpha ( p_\alpha - p_\alpha^0 ) ]        \quad,
\]
fluid fraction in the reservoir (porosity):
\[
  \phi  \approx \phi_0 \left[ 1 + c_f ( p - p_0 ) \right]     \quad,
\]
and permeability tensor
\[
  \bm{K}_\alpha = \bm{K} k_{r\alpha}      \quad,
\]
where $\bm{K}$ is the absolute (intrinsic) permeability tensor and $k_{r\alpha}$ is the relative permeability of that phase.

Consider the saturation restrictions:
\[
  1 = \sum_\alpha S_\alpha      \quad,
\]
and cappilary pressure for a two-phase flow system
\[
  p_o = p_w + P_{cow}   \quad.
\]

Define the unknonwns refering to toal quantities:
{  % Symbol definitions for the section
  \newcommand{\Gammaw}{{\Gamma_{\text{w}}}}
  \newcommand{\Gammao}{{\Gamma_{\text{o}}}}
  \newcommand{\zetaalpha}{\zeta_\alpha}
  \newcommand{\zetaw}{\zeta_\text{w}}
  \newcommand{\zetat}{{\zeta_{\text{t}}}}
  \newcommand{\qalpha}{q_\alpha}
  \newcommand{\qw}{{q_{\text{w}}}}
  \newcommand{\qt}{{q_{\text{t}}}}
  \newcommand{\uualpha}{\uu_\alpha}
  \newcommand{\uuw}{\uu_\text{w}}
  \newcommand{\uut}{{\uu_{\text{t}}}}
  \newcommand{\qcq}{\ ,\ }
  \newcommand{\Omegawell}{\Omega_{\text{well}}}
  \newcommand{\nn}{\mathbf{n}}
\[\begin{array}{ c c c }
  \displaystyle    \zetat = \sum_{\alpha\in\text{w,o}} \zetaalpha  \quad,  & 
  \displaystyle    \uut = \sum_{\alpha\in\text{w,o}} \uualpha      \quad,  &
  \displaystyle    \qt = \sum_{\alpha\in\text{w,o}} \qalpha  
\end{array}\]

The problem statement is:

Find $p_\alpha$ and $S_\alpha$ for $\alpha \in \{\text{w, o}\}$, such that 

\[\begin{array}{ l l }
  \displaystyle
   0 = \dot{\zetat} + \nabla\cdot\uut- \qt                        & \quad\quad \text{in } \Omega        \\ 
   0 = \dot{\zeta_w} + \nabla\cdot\uu_w - q_w                     & \quad\quad \text{in } \Omega        \\ 
   1 = \sum_\alpha S_\alpha                                       & \quad\quad \text{in } \Omega        \\
   0 = \nabla p_w                                                 & \quad\quad \text{in } \Gammao      \\
   1 = S_o                                                        & \quad\quad \text{in } \Omega \times t=0  \\
   q_\alpha = \hat{q}_\alpha                                      & \quad\quad \text{in } \Omegawell  \quad \text{(wells interface in sink terms)}    \\
\end{array}\]


Define the bilinear operator as
\[
(v,w)_\Omega = \int_\Omega v\ w\ d\Omega
\]
where $\Omega$ is the integration domain. 
If suppressed, $\Omega$ is the domain, otherwise it must be specified.
Between two vectors, the operator considers the dot product as
\[
(\vv,\ww)_\Omega = \int_\Omega \vv \cdot \ww\ d\Omega
\]

Now we describe the weak discrete form for the problem.

\vspace{0.5cm}
Find $p_\text{w}^h$ and $S_\text{w}^h$ for $\alpha \in \{\text{w, o}\}$, $\forall w^h \in \mathcal{W}^h$, $\forall v^h \in \mathcal{V}^h$
\begin{align}
  \displaystyle
  0 &= ( w^h \qcq \dot{\zetat} ) + ( w^h \qcq \nabla \cdot \uut ) - ( w^h \qcq \qt )   \label{eqn:weak1}\\
  0 &= ( v^h \qcq \dot{\zetaw} ) + ( v^h \qcq \nabla \cdot \uuw ) - ( v^h \qcq \qw )   \label{eqn:weak2} 
\end{align}
subject to the boundary constraints and initial conditions previously defined.
The second term on the RHS requires attention because it has the second order derivative in $p$.
Let's work it separately:
\newcommand{\OmegaI}{{\Omega_{\text{I}}}}
\newcommand{\GammaI}{{\Gamma_{\text{I}}}}
\newcommand{\GammaO}{{\Gamma_{\text{O}}}}
\newcommand{\GammaIe}{{\Gamma_{\text{I}}^e}}
\begin{align}
  ( w^h \qcq \nabla \cdot \uu ) &=  \sum_E ( w^h \qcq \nabla \cdot \uu )_{\Omega^e}  \nonumber\\
                                &=( w^h \qcq \uu \cdot \nn )_\GammaO + ( w^h \qcq \uu \cdot \nn )_\GammaI - ( \nabla w^h \qcq \uu )_\OmegaI
\end{align}
where $\OmegaI$ represent the element interiors, $\GammaI = \Gamma^e \setminus \GammaO$ is the internal element edges
(i.e. the internal skelethon of the mesh), $\Gamma^e=\GammaI\cup\GammaO$ is all the element edges, and $\GammaO=\Gamma^e\cap\Gamma$ is the
external boundaries of the domain.

Recall the identity (see appendix):
\def\jmp#1{\llbracket #1 \rrbracket}
\def\sjmp#1{\left[  #1  \right]}


\def\avg#1{\{ #1 \}}
\begin{align}
  ( w \qcq \uu \cdot \nn)_\GammaI \equiv
  ( \jmp{w} \qcq \avg{\uu} )_\GammaI + ( \avg{w} \qcq \jmp{\uu} )_\GammaI
\end{align}
where
\[\begin{array}{ c c }
  \jmp{\uu}    =  \uu^+\cdot\nn^+  + \uu^-\cdot\nn^-      &  \qquad
  \avg{\uu}    = 0.5 \  (\ \uu^+         +\ \uu^- \ )  \nonumber\\
  \jmp{w}      =  w^+ \cdot \nn^+ + w^- \cdot \nn^-      &  \qquad
  \avg{w}      = 0.5 \  (\ w^+         +\ w^- \ )       
\end{array}\]
Recall the boundary condition $\uu=0$ in $\GammaO$. Then:
\[\begin{array}{ c c c c }
  ( w^h \qcq \nabla \cdot \uu ) &= 
                                    ( \jmp{w^h} \qcq \avg{\uu} )_\GammaI 
                                & + ( \jmp{\uu} \qcq \avg{w^h} )_\GammaI 
                                & - ( \nabla w^h \qcq \uu )_\OmegaI
\end{array}\]
that is our final expression for this term.

Now we can rewrite \eqref{eqn:weak1} and \eqref{eqn:weak2} using the above identity as:
\newcommand{\blue}[1]{\textcolor{blue}{#1}}
\newcommand{\red}[1]{\textcolor{red}{#1}}
\[\begin{array}{ c c c c c c }
  0 &= ( w^h \qcq \dot{\zetat} ) 
            \quad & \blue{  + ( \jmp{w^h} \qcq \avg{\uut} )_\GammaI  }
            \quad & \blue{  + ( \avg{w^h} \qcq \jmp{\uut} )_\GammaI   }
            \quad & \blue{  - ( \nabla w^h \qcq \uut )_\OmegaI  }
            \quad &-  ( w^h \qcq \qt )   \\
    &+ ( v^h \qcq \dot{\zetaw} ) 
            \quad & \blue{  + ( \jmp{v^h} \qcq \avg{\uuw} )_\GammaI  }
            \quad & \blue{  + ( \avg{v^h} \qcq \jmp{\uuw} )_\GammaI   }
            \quad & \blue{  - ( \nabla v^h \qcq \uuw )_\OmegaI  }
            \quad &-  ( v^h \qcq \qw )   \\
\end{array}\]

\section{Local mass conservation}
The formulation is conservative iff $\jmp{\uualpha}=0$, that is
\[
  0 = \uu^+ \cdot \nn^+ + \uu^- \cdot \nn^- 
\]
We want to set the constraint weekly in both element boundaries sharing the edge:
\begin{align}
  0 &= (\ w^+ \qcq \uu^+ \cdot \nn^+\ )_{\GammaIe} + (\ w^+ \qcq \uu^- \cdot \nn^-\ )_{\GammaIe}  \\
  0 &= (\ w^- \qcq \uu^+ \cdot \nn^+\ )_{\GammaIe} + (\ w^- \qcq \uu^- \cdot \nn^-\ )_{\GammaIe}       \quad.
\end{align}
or alternatively
\begin{align}
  0 &= (\ w^+ \nn^+ \qcq \uu^+ \ )_{\GammaIe} - (\ w^+ \nn^+ \qcq \uu^- \ )_{\GammaIe}  \\
  0 &=  (\ w^- \nn^- \qcq \uu^- \ )_{\GammaIe} -(\ w^- \nn^- \qcq \uu^+ \ )_{\GammaIe}  \\
  &  \rightarrow \quad 0 = ( \jmp{w} \qcq \uu^+ - \uu^- ) 
  \quad.
\end{align}
We'll stick with the first solution for now. The second could help collapsing some terms later (we can try later).
For the set of internal element interfaces of the mesh $\GammaI$, we can write
\begin{align}
  0 = (\ \sjmp{w} \qcq \jmp{\uu} \ )_\GammaI
\end{align}
where we define the scalar jump operator as
\[
  \sjmp{w} = w^+ - w^-   \quad.
\]
Note the difference between the vectorial jump $\jmp{w}$ (a summation scaled by the respective 
normals and thus independent on the assignment of $+$ and $-$ to the neighboring elements) 
and the scalar jump $\sjmp{w}$, that is a difference results in a scalar. In the latter, 
the sign of the result depends on the element assignment.

The above constraints at the element boundaries follow the interior penalty method 
\parencite{douglas1976, wheeler1978}:
\newcommand{\Jt}{{J_{\text{t}}}}
\begin{align}
  J_\alpha(w,\uu) &= \int_\GammaI \gamma \sjmp{w} \jmp{\uualpha} d\Gamma  \\
                  &= (\ \gamma \sjmp{w} \qcq \jmp{\uualpha} \ ) \rightarrow  0
\end{align}
and
\begin{align}
  \displaystyle  \Jt &= \sum_{\alpha\in\text{w,o}} J_\alpha       \quad,
\end{align}
where $\gamma$ is a penalty term. 

The formulation becomes
\begin{align}
  & \forall w^h \in \mathcal{W}^h, \forall v^h \in \mathcal{V}^h \nonumber & \\
  & \begin{array}{ cccccccc }
  0 &= ( w^h \qcq \dot{\zetat} ) 
            \quad &  + ( \jmp{w^h} \qcq \avg{\uut} )_\GammaI  
            \quad &  + ( \avg{w^h} \qcq \jmp{\uut} )_\GammaI   
            \quad &  - ( \nabla w^h \qcq \uut )_\OmegaI  
            \quad &-  ( w^h \qcq \qt ) 
            \quad &+ ( \gamma \sjmp{w^h} \qcq \jmp{\uut} )_\GammaI
                  &\\
    &+ ( v^h \qcq \dot{\zetaw} ) 
            \quad &  + ( \jmp{v^h} \qcq \avg{\uuw} )_\GammaI  
            \quad &  + ( \avg{v^h} \qcq \jmp{\uuw} )_\GammaI   
            \quad &  - ( \nabla v^h \qcq \uuw )_\OmegaI  
            \quad &-  ( v^h \qcq \qw )   
            \quad &+ ( \gamma \sjmp{v^h} \qcq \jmp{\uuw} )_\GammaI
                  &,
          \end{array}  
  &
        \end{align}


\section{Finite Volume Formulation}


\appendix

\section{Discontinuous identity}

An alternative derivation:
\begin{align}
  ( \jmp{w} \qcq \{\uu\} )_\GammaI + ( \jmp{\uu} \qcq \{ w \} )_\GammaI &= 0.5 \times \\
    &\quad\quad (w^+\nn^+ \qcq \uu^+)                     \quad+\quad                      (\uu^+ \cdot\nn^+\qcq w^+ ) \\
    &\quad\quad \textcolor{red}{(w^-\nn^+ \qcq \uu^+)}    \quad+\quad   \textcolor{red}{   (\uu^+ \cdot\nn^-\qcq w^- )   } \\
    &\quad\quad \textcolor{blue}{(w^+\nn^- \qcq \uu^-)}   \quad+\quad   \textcolor{blue}{  (\uu^- \cdot\nn^+\qcq w^+ )   } \\
    &\quad\quad (w^-\nn^- \qcq \uu^-)                     \quad+\quad                      (\uu^-\cdot \nn^- \qcq w^- ) \\
    &= 0.5 \times \\
    &\quad\quad (w^+\nn^+ \qcq \uu^+)                     \quad+\quad                      (\uu^+ \cdot\nn^+\qcq w^+ ) \\
    &\quad\quad (w^-\nn^- \qcq \uu^-)                     \quad+\quad                      (\uu^-\cdot \nn^- \qcq w^- ) \\
    &= (w \qcq \uu\cdot\nn)_\GammaI
\end{align}
where $\GammaI$ subscript was supressed along the derivation for clarity.
For implementation purposes, it is convenient to use iterators ($D$, $E$), 
\begin{align}
  ( \jmp{w}  &\qcq \{\uu\} )_\GammaI + ( \jmp{\uu} \qcq \{ w \} )_\GammaI =  \\
                                &= 0.5\times \sum_{D=1}^{2}\sum_{E=1}^{2} 
                                             ( w^D \nn^D \qcq \uu^E )_\GammaI
                                           + ( \uu^D \cdot \nn^D \qcq w^E )_\GammaI\\
                                &=  \sum_{D=1}^{2}( w^D ,\quad \uu^D \cdot \nn^D )_{\GammaI} 
\end{align}
where $D,E=1$ represents the $+$ side and $D,E=2$ represents the $-$ side.

Similarly,
\begin{align}
  ( \jmp{w} \qcq \jmp{\uu} ) = \sum_{D=1}^{2}\sum_{E=1}^{2} ( w^D \nn^D \qcq \uu^E \cdot \nn^E )
\end{align}
}

\newpage
\printbibliography
\end{document}
