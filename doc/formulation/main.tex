\documentclass[11pt]{article}
\usepackage[margin=1.5cm, top=1cm]{geometry}
\usepackage{amsmath,amssymb,scalerel}
\usepackage{bbm}
\usepackage{stmaryrd} % for symbols like \llbracket
\usepackage{mathabx}
\usepackage{bm}
\usepackage{xcolor}
\usepackage{siunitx}
\usepackage{mathtools}
\usepackage{mathrsfs}
\usepackage[scr=rsfs,cal=boondox]{mathalfa}

\usepackage{array}
\usepackage{graphicx}

\usepackage[style=apa,backend=biber]{biblatex}
\addbibresource{../ref.bib}


\newcolumntype{A}{c@{\hspace{.5cm}}}
\newcolumntype{B}{c@{\hspace{2cm}}}
\newcolumntype{L}{>$l<$} 
\setlength{\jot}{6pt}
\renewcommand{\arraystretch}{1.5}

\title{Crk3 formulation}
\author{Renato Poli}
\begin{document}
\maketitle
\hrule

% some accelarators
\def\u{\bm{u}}
\def\n{\bm{n}}
\def\K{\bm{K}}
\def\F{\bm{F}}
\def\I{\bm{I}}
\def\U{\bm{U}}
\def\w{\bm{w}}
\def\W{\bm{W}} 
\def\N{\bm{N}}
\def\x{\bm{x}}
\def\X{\bm{X}}
\def\cF{\mathcal{F}}
\def\CC{\mathbb{C}}
\def\cc{\mathbbm{c}}
\newcommand{\cof}{\mathop{\mathrm{cof}}}
\newcommand{\adj}{\mathop{\mathrm{adj}}}
\newcommand{\tr}{\mathop{\mathrm{tr}}}
\def\eps{\bm{\varepsilon}}
\def\jmp#1{\llbracket #1 \rrbracket}
\def\ss#1#2#3{\tiny\substack{#1\\#2\\#3}}
\def\vepbd{\dot{\bar{\varepsilon}}}  % varepsilon, bar and dot

\section{Fracture storage}

\[
  \zeta_f = \frac{\delta a_f}{a_f} 
      + c_f \delta p
      - \beta_f \delta T \\
\ ,\]
and we need the term
\begin{align}
  \label{eqn:zeta_variational}
  ( \tilde{\psi}, a_f \dot{\zeta}_f )_{\tilde{\Omega}} =  
    ( \tilde{\psi}, \delta \dot{a}_f) 
  + ( \tilde{\psi}, a_f c_f \delta \dot{p})
  - ( \tilde{\psi}, a_f \beta_f \delta \dot{T} )
\ ,
\end{align}
where 
\begin{align*}
  a_f &= \jmp{\u} \cdot \n^+ \ =
         ( \u^+-\u^- ) \cdot \n^+ = 
         \sum_{d=1}^2 u^d_k\ n^d_k \ =
         u^d_k\ n^d_k \ .
\end{align*}
where $d=0$ for the element in the $+$ side of the fracture and $d=1$ for the element in the $-$ side.

Opening the terms in \eqref{eqn:zeta_variational} into coefficients (assuming $p$ to be a step constant):
\begin{align*}
  + ( \tilde{\psi}, \delta \dot{a}_f) \Delta t : & \\
               \K \ss{\beta\gamma}{pd}{0k} & \quad = \quad
                   \tilde{\psi}^\beta\   \quad
                   (-1)^d\               \quad
                   \phi\ss{\gamma}{d}{}  \quad
                   n\ss{}{+}{k}          \\
               \F \ss{\beta}{p}{0} &   \quad = \quad
                   \tilde{\psi}^\beta\   \quad
                   (-1)^d\               \quad
                   \phi\ss{\gamma}{d}{}  \quad
                   n\ss{}{+}{k}          \quad
                   \check{u} \ss{\gamma}{d}{k} 
\ \ , \\      
 + ( \tilde{\psi}, a_f c_f \delta \dot{p}) \Delta t : & \\
               \K \ss{\beta\gamma}{pd}{0k} & \quad = \quad
                   c_f                   \quad
                   \tilde{\psi}^\beta\   \quad
                   (-1)^d\               \quad
                   \phi\ss{\gamma}{d}{}  \quad
                   n\ss{}{+}{k}          \quad 
                   \Delta p               
\ \ , \\      
 - ( \tilde{\psi}, a_f \beta_f \delta \dot{T} ) \Delta t : & \\
               \K \ss{\beta\gamma}{pd}{0k} & \quad = \quad
                   \beta_f               \quad
                   \tilde{\psi}^\beta\   \quad
                   (-1)^{d+1}            \quad
                   \phi\ss{\gamma}{d}{}  \quad
                   n\ss{}{+}{k}          \quad 
                   \Delta T               
\ \ .
\end{align*}
where $\check{u}$ refers to the (known) solution in the previous timestep, and $c_f= \frac{1}{K_f}$.
$\Delta p$ and $\Delta T$ are the delta of $p$ and $T$ from the previous timestep to the current.


\newpage
\section{Derivatives for nonlinear solution}

Both pressure and displacements are variables of the system.
We need to derive some non-linear term for Newton's method.
Let:
\begin{align*}
 f = \left( \tilde{\psi}, \quad
             a_f\ 
             c_f\  
             \delta \dot{p} \right) \Delta t = 
 \sum_{d=1}^2 \left( \tilde{\psi}, \quad
                         c_f\ 
                         n^d_k\ 
                         u^d_k\ 
                         \delta \dot{p}
      \right) \Delta t 
      \quad .
\end{align*}
where $d=0$ for the element in the $+$ side of the fracture and $d=1$ for the element in the $-$ side.
Then,the partial derivatives are
\begin{align*}
  \frac{\partial f}{\partial p} = 
       \sum_{d=1}^2 \left( \tilde{\psi}, \quad
                               c_f\ 
                               n^d_k\ 
                               u^d_k 
            \right) \Delta t 
      \quad ,
\end{align*}
and
\begin{align*}
  \frac{\partial f}{\partial u\ss{}{d}{k}} = 
       \left( \tilde{\psi}, \quad
                         c_f\ 
                         n^d_k\ 
                         \delta \dot{p}\ 
            \right) \Delta t 
        \quad .
\end{align*}


\newpage
\section{Discontinuous galerkin for pressure}

Reference: (Ruijie Liu, 2004) - PhD dissertation

Our equation:
\begin{align*}
  \alpha \nabla \cdot \dot{u}
  + S_\epsilon\ \dot{p} 
  - \nabla \cdot ( \check{\kappa} \nabla p) 
  = 0
\end{align*}
Weak formulation:
\begin{align*}
  ( w, \alpha \nabla \cdot \dot{u} )
  + (w, S_\epsilon\ \dot{p} )
  - ( w, \nabla \cdot ( \check{\kappa} \nabla p)  )
  = 0 \quad\quad \forall w
\end{align*}
Lets focus our attention to the last term. Integrate by parts:
\begin{align*}
  - ( w, \nabla \cdot ( \check{\kappa} \nabla p)  ) =
        ( \nabla w, \check{\kappa} \nabla p )
        - ( w, \check{\kappa} \nabla p \cdot \n )_{\Gamma}
\end{align*}
When we sum element by element, the last term in $\Gamma$ results in the edge skeleton $\Gamma_i$ 
and the outside boundary $\Gamma_p$.

The full equation becomes a summation of the following:
\begin{align*}
  ( w, \alpha \nabla \cdot \dot{u} )_{\Omega_E}
  + (w, S_\epsilon\ \dot{p} )_{\Omega_E}
  + ( \nabla w, \check{\kappa} \nabla p )_{\Omega_E}
  - ( w, \check{\kappa} \nabla p \cdot \n )_{\Gamma_i}
  - ( w, \check{\kappa} \nabla p \cdot \n )_{\Gamma_p}
  = 0 \quad\quad \forall w
\end{align*}
where $\Omega_E$ is the element interiors

We observe the following identity:
\begin{align}
  \label{eqn:dg_term}
  - ( w, \check{\kappa} \nabla p \cdot \n )_{\Gamma_i} =
  - ( \jmp{w}, \check{\kappa} \{\nabla p\})_{\Gamma_i} 
  - ( \check{\kappa} \jmp{\nabla p}, \{w\})_{\Gamma_i} 
\end{align}
where
\begin{align*}
  \jmp{w} &= w^+ \n^+ + w^- \n^- \\
  \{w\} &=  0.5 \times ( w^+ + w^- ) 
\end{align*}
and similarly
\begin{align*}
  \jmp{\nabla p} &= \nabla p^+ \cdot \n^+ + \nabla p^- \cdot \n^- \\
  \{\nabla p\} &=  0.5 \times ( \nabla p^+ + \nabla p^- ) 
\end{align*}
Define $d=0$ for $+$ and $d=1$ for $-$ and similarly $e=0$ for $+$ and $e=1$ for $-$. 
The above can be written as summations in $d$ :
\begin{align*}
  \jmp{w} &= \sum_d w^d n^d_k   \\
  \{w\}    &= 0.5 \times \sum_d w^d \\
  \jmp{\nabla p} &= \sum_e p_{,k}^e n_k^e   \\
  \{\nabla p\}    &= 0.5 \times \sum_e p_{,k}^e
\end{align*}
Replace the above in \eqref{eqn:dg_term} suppressing the summation sign in $d$ and $e$:
\begin{align*}
  - ( w, \check{\kappa}\ \nabla p \cdot \n )_{\Gamma_i} =
  - 0.5 \times \left[ 
        \left( w^d\ n^d_k,\quad \check{\kappa}\ p_{,k}^e \right)_{\Gamma_i} 
      \quad+\quad 
        \left( \check{\kappa}\ p_{,k}^e\ n^e_k ,\quad  w^d \right)_{\Gamma_i} 
    \right]
\end{align*}

Finally, we add an interior penalty term:
\begin{align*}
  \frac{\delta_p}{|s|} \left( \jmp{p} , \jmp{w} \right)_{\Gamma_i}
\end{align*}

The final expression is:
\begin{align*}
  & ( w, \alpha \nabla \cdot \dot{u} )_{\Omega_E}
  + (w, S_\epsilon\ \dot{p} )_{\Omega_E}
  + ( \nabla w, \check{\kappa} \nabla p )_{\Omega_E}
  - ( w, \check{\kappa} \nabla p \cdot \n )_{\Gamma_p}\\[10pt]
  & - ( \jmp{w}, \check{\kappa} \{\nabla p\})_{\Gamma_i} 
  - ( \check{\kappa} \jmp{\nabla p}, \{w\})_{\Gamma_i} \\[10pt]
  & + \frac{\delta_p}{|s|} \left( \jmp{p} , \jmp{w} \right)_{\Gamma_i}
  = 0 \quad\quad \forall w
\end{align*}

\newpage
\section{An identity}
\begin{align*}
  ( w, \u \cdot \n )_{\Gamma_i}
              &= ( w^+, \u^+ \cdot \n_+ )_{\Gamma_i} + ( w^-, \u^- \cdot \n^- )_{\Gamma_i}\\
              &= ( \jmp{w} , \{\u\} ) + ( \jmp{\u}, \{ w \} ) \\
              &= 0.5\times ( w^D ,\quad \u^E \cdot \n^E )_{\Gamma_i}        &+&
                 0.5\times ( \u^D ,\quad w^E\ \n^E )_{\Gamma_i} \\
              &= 0.5\times (-1)^E ( w^D ,\quad \u^E \cdot \n^+ )_{\Gamma_i} &+&
                 0.5\times (-1)^E ( \u^D ,\quad w^E\ \n^+ )_{\Gamma_i} \\
              &= 0.5\times (-1)^E ( w^D ,\quad \u^E \cdot \n^+ )_{\Gamma_i} &+&
                 0.5\times (-1)^D ( w^D ,\quad \u^E\ \n^+ )_{\Gamma_i} \\
              &= 0.5\times \left[ (-1)^E + (-1)^D \right] ( w^D ,\quad \u^E \cdot \n^+ )_{\Gamma_i} \\
              &= \delta^{DE} (-1)^D ( w^D ,\quad \u^E \cdot \n^+ )_{\Gamma_i} \\
              &= (-1)^D ( w^D ,\quad \u^D \cdot \n^+ )_{\Gamma_i} \\
              &= ( w^D ,\quad \u^D \cdot \n^D )_{\Gamma_i} \\
\end{align*}

An alternative derivation:
\begin{align*}
  & ( \jmp{w} , \{\u\} ) + ( \jmp{\u}, \{ w \} ) = \\
              &= 0.5\times ( w^D ,\quad \u^E \cdot \n^E )_{\Gamma_i} +
                 0.5\times ( \u^D ,\quad w^E\ \n^E )_{\Gamma_i}\\
    &= \\
    &\quad\quad (w^+, \u^+\cdot\n^+)    \quad+\quad   (\u^+, w^+ \n^+) \\
    &\quad\quad \textcolor{red}{(w^-, \u^+\cdot\n^+)}    \quad+\quad   \textcolor{red}{(\u^+, w^- \n^-)} \\
    &\quad\quad \textcolor{blue}{(w^+, \u^-\cdot\n^-)}   \quad+\quad   \textcolor{blue}{(\u^-, w^+ \n^+)} \\
    &\quad\quad (w^-, \u^-\cdot\n^-)    \quad+\quad   (\u^-, w^- \n^-) \\
    &= \\
    &\quad\quad (w^+, \u^+\cdot\n^+)    \quad+\quad   (\u^+, w^+ \n^+) \\
    &\quad\quad (w^-, \u^-\cdot\n^-)    \quad+\quad   (\u^-, w^- \n^-) \\
    &= \\
    &\quad\quad ( w^D ,\quad \u^D \cdot \n^D )_{\Gamma_i} \\
\end{align*}

\newpage
\section{Viscoelasticity (Creep)}

We follow the workflow by \textcite{kumar2021} for the formulation of the viscoelastic model for salt.
According to \parencite{carter1993}, the strain of a solid is
\begin{align}
  \eps=\eps^e+\eps^p+\eps^t+\eps^s+\eps^a \quad,
\end{align}
where the superscripts stand for elastic, plastic, transient (or primary or decelerating creep), steady-state (or secondary creep) and accelerating (or tertiary creep), respectively.

\subsection{Multiaxial creep - development}

\begin{align*}
 & \sigma_e^2= \frac{3}{2} \tau_{ij} \tau_{ij} \\
 & \tau_{ij} = \sigma_{ij} - \frac{1}{3} \sigma_{kk} \delta_{ij}
\end{align*}
Now we want to develop the equality:
\begin{align*}
  \dot{\varepsilon}_{ij}^s = \frac{\partial \sigma_e}{\partial \sigma_{ij}} \dot{\bar{\varepsilon}} = \frac{3}{2} \frac{\tau_{ij}}{\sigma_e}\ \dot{\bar{\varepsilon}}^s 
\end{align*}
Hence:
\begin{align*}
 2 \sigma_e \ \partial\sigma_e &= 3 \tau_{ij} \ \frac{\partial\tau_{ij}}{\partial \sigma_{kl}} \partial \sigma_{kl}  \\
                               &= 3 \tau_{ij} \ \left[\frac{\partial\sigma_{ij}}{\partial \sigma_{kl}} - \frac{1}{3} \frac{\partial\sigma_{mm}}{\partial\sigma_{kl}}\right]\ \partial \sigma_{kl}  \\
                               &= 3 \tau_{ij} \ \left[\delta_{ik}\delta_{jl} - \frac{1}{3} \delta_{kl}\delta_{ij}\right]\ \partial \sigma_{kl}  \\
                               &= 3 \tau_{kl}\partial\sigma_{kl} - \tau_{ii}\partial\sigma_{kl}\delta_{kl} \\
                               &= 3 \tau_{kl}\partial\sigma_{kl} 
\end{align*}
because $\tau_{ii}=0$. Hence:
\begin{align*}
  \frac{\partial\sigma_e}{\partial\sigma_{kl}} = \frac{3}{2} \frac{\tau_{kl}}{\sigma_e}
\end{align*}
%
%
%
%
%
\subsection{Carter model}

The starting point is Carter's model \parencite{carter1993}.
We write in the normalized way so that the constants have rational units \parencite{dusseault1993}, and dimensionless parameters.
The expression for the creep model is the following power-law for steady-state creep is obtained from the lab for uniaxial strain as
\begin{align}
  \dot{\bar{\varepsilon}}^{s} =
      \varepsilon_0\  
      \exp\left({-\frac{Q}{RT}}\right)\ 
      \left(\frac{\sigma_e}{\sigma_0}\right)^n 
\end{align}
where $\sigma_0$ is a normalization stress arbitrary in some sense as it is a fitting parameter together with $\varepsilon_0$.
It can be seen as a critical stress for transition of regimes (when the quotient approaches $1$, we can continuously change the exponent $n$). See \parencite{dusseault1987, poiate2012} .
Table \ref{tab:carter_symbols} describes the variables.

Now we express the creep flow rule in tensor form under multiaxial stress conditions \parencite{xu2018,kraus1980,hyde2014}, as:
\begin{align}
  \dot{\varepsilon}_{ij}^s = \frac{\partial \sigma_e}{\partial \sigma_{ij}} \dot{\bar{\varepsilon}} = \frac{3}{2} \frac{\tau_{ij}}{\sigma_e}\ \dot{\bar{\varepsilon}}^s 
\end{align}
where $\sigma_e$ is an effective stress measure (here assumed Von Misses) and $\tau_{ij}$ is the stress deviator tensor.
\begin{align}
  \sigma_e^2 = \frac{3}{2}\ \tau_{ij} \tau_{ij} \\
  \tau_{ij} = \sigma_{ij} - \frac{1}{3} \sigma_{kk} \delta_{ij}
\end{align}
Equivalent expressions for $\sigma_e$ are
\begin{align}
  & \sigma_e^2=3\ J_2 \quad\quad J_2=0.5\times \tau_{ij}\tau_{ij} \\
 & \sigma_e^2 = 0.5\times\left[ 
        (\sigma_{xx} - \sigma_{yy})^2  + (\sigma_{yy} - \sigma_{zz})^2  + (\sigma_{zz} - \sigma_{xx})^2  +
        6\ ( \sigma_{xy}^2 + \sigma_{yz}^2 + \sigma_{xz}^2 )
    \right] \\
 &   \sigma_e^2 = \frac{3}{2} \sigma_{ij} \sigma_{ij} - \frac{1}{2} (\sigma_{kk}^2 )
\end{align}
where $J_2$ is the second invariant of the stress deviator tensor.
We can use Carter's law to write
\begin{align}
  \dot{\varepsilon}_{ij}^s =
          \frac{3}{2}\ \varepsilon_0\ 
          \exp\left({-\frac{Q}{RT}}\right)\ 
          \left(\frac{\sigma_e}{\sigma_0}\right)^n\ 
          \left(\frac{\tau_{ij}}{\sigma_e}\right)
\end{align}
An alternative writing for this equation, as e.g. in \parencite{kumar2021}, is 
\begin{align}
  \dot{\varepsilon}_{ij}^s =
          \frac{3}{2}\ A\ 
          \exp\left({-\frac{Q}{RT}}\right)\ 
          \sigma_e^{n-1}\ 
          \tau_{ij}
\end{align}
where $A=\varepsilon_0/\sigma_0^n$. However, the units of $A$ are $MPa^{-n}$ and $A$ and $n$ are mutually dependent.
The use of $\sigma_0$ is convenient for decoupling and different creep modes can be calibrated using only $n$, for example.
Alternativelly, we can use the strategy by \parencite{dusseault1987}, with a critical stress and sharp transitions between modes..
\begin{align}
  A= \frac{\varepsilon_0}{\sigma_0^n}
          \exp\left({\frac{Q}{RT}}\right)\ 
\end{align}

\begin{table}[htbp]
\centering
\begin{tabular}{ c c c }
\hline Variable & Unit & Description \\
\hline $R$ & J mol$^{-1}$ K$^{-1}$  & Universal gas constant ($R=8.3144$)\\
\hline $T$ & K                      & Temperature \\
\hline $Q$ & J mol$^{-1}$          & Apparent activation energy \\
\hline $n$ & $-$                     & Material constant \\
\hline $\varepsilon_0$ & $-$     & Material constant \\
\hline $\sigma_0$ & MPa           & Material constant \\
\hline $\sigma_e$ &MPa                 & Differential stress ($\sigma_1-\sigma_3$), or von mises, or Tresca. \\
\hline
\end{tabular}
\caption{Symbols in Carter model}
\label{tab:carter_symbols}
\end{table}

\begin{table}[htbp]
\centering
\begin{tabular}{ c c }
\hline Variable & Value \\
\hline $Q$      & $51.6 \times 10^{-3}$ J mol$^{-1}$  \\ 
\hline $R$      & $8.3144$ J mol$^{-1}$ K$^{-1}$ \\ 
\hline $A$      & $8.1\times 10^{-5}$ MPa$^{-n}$ s$^{-1}$  \\ 
\hline
\end{tabular}
\caption{Values for initial testcase, to compare with results from \textcite{carter1993} and \textcite{kumar2021}.}
\label{tab:carter_values_0}
\end{table}

\subsection{Creep in small strains}
Let $\eps^s$ be the steady state creep history, so that
\begin{align*}
  \eps^s (t+\Delta t) &= \eps^s(t) + \Delta \eps^s(t+\theta) \quad, \\
  \Delta \eps^s &= \dot{\eps}(t)\ \Delta t
\end{align*}
is an explicit timetepping. We can calculate $\dot{\eps}^s$ from the previous section.

The problem statement is:

Find $\u\in\mathcal{V}$ such that $\forall \w\in\mathcal{W}$
\begin{align*}
  & (w_{i,j},\ \sigma_{ij}) - (w_i,\ \mathcal{f}_i) - (w_i,\ \mathcal{h}_i)_{\Gamma_h} = 0 \\
  & \sigma_{ij} = \CC_{ijkl}\ \varepsilon^e_{kl} \\
  & \varepsilon^e_{kl} = \varepsilon_{kl} - \varepsilon^s_{kl} \\
  & \varepsilon_{kl} = 0.5\times(u_{k,l} + u_{l,k}) \\
  & u_i = \mathcal{g}_i                                & \text{on } \Gamma_g \\
  & \sigma_{ij}\ n_j = \mathcal{h}_i                     & \text{on } \Gamma_h 
\end{align*}
where $\Omega$ is the domain of the the bilinear operators whenever ommitted.
$u_i$ agregates the unknown and known (Dirichlet) degrees of freedom, as usual.

The creep term is integrated in time and added to the equilibrium equation as
\begin{align*}
  & (w_{i,j},\ \CC_{ijkl}\ u_{k,l}) - (w_{i,j},\ \CC_{ijkl}\ \varepsilon^s_{kl}) - (w_i,\ \mathcal{f}_i) - (w_i,\ \mathcal{h}_i)_{\Gamma_h} = 0 
\end{align*}

\subsection{Munson Dawson - transient creep}
This section is based on \parencite{reedlunn2018, munson1999, cheng2024}, that follow previous work at Sandia \parencite{munson1982,munson1989}.

Define the total plastic strain rate as
\begin{align}
  \vepbd_p = \vepbd_{ss} + \vepbd_{tr} = F\ \vepbd_{ss}
\end{align}
where the steadystate creep is
\begin{align}
  \vepbd_{ss} = \varepsilon_0\ \exp{\frac{-Q}{RT}}\ \bar\sigma^n
\end{align}
where the normalized stress is 
\begin{align}
  \bar\sigma=\frac{\sigma_e}{\sigma_0}
\end{align}
and $\sigma_e$ is the deviatoric (or equivalent, or von mises, or Hosford) stress.

The transient creep is
\begin{align}
  \vepbd_{tr} = (F-1)\ \vepbd_{ss}
\end{align}
where
\begin{align}
  \def\arraystretch{2}
  F =
  \left\{
    \begin{array}{ll}
      \exp \left( \omega_w\ \zeta^2 \right) & \quad,\quad \zeta > 0 \ (\varepsilon_{tr}\leq\varepsilon^*_{tr})\\
      \exp \left( - \omega_r\ \zeta^2 \right) & \quad,\quad \zeta < 0 \ (\varepsilon_{tr}>\varepsilon^*_{tr})\\
    \end{array}
  \right.
\end{align}
where
\begin{align}
  \zeta &= 1 - \frac{\varepsilon_{tr}}{\varepsilon^*_{tr}} \\
  \omega_w &= \alpha_w + \beta_w\ \log_{10} \bar\sigma \\
  \omega_r &= \alpha_r + \beta_r\ \log_{10} \bar\sigma
\end{align}
and
\begin{align}
  \varepsilon_{tr}^* = K\ \exp(c\,T)\ \bar\sigma^m
\end{align}

The above expressions are the ones given by MD.
Now we want to simplify to ease our implementation and analysis.
\[\begin{array}{ccl}
  \vepbd_{ss} = \varepsilon_0\ \exp{\frac{-Q}{RT}}\ \bar\sigma^n
  && 
    \varepsilon_{tr}^* = K\ \exp(c\,T)\ \bar\sigma^m 
  \\[10pt]
    \zeta = 1 - \frac{\varepsilon_{tr}}{\varepsilon^*_{tr}} 
  & \rightarrow \qquad\qquad  & 
    \begin{cases}
      \ \zeta > 0 \quad\rightarrow\quad    (\alpha \ ,\  \beta) = + \ ( \alpha_w \ ,\  \beta_w )  \\
      \ \zeta \leq 0 \quad\rightarrow\quad (\alpha \ ,\  \beta) = - \ ( \alpha_r \ ,\  \beta_r ) \\
    \end{cases}
  \\[15pt]
      F = e^{\alpha\,\zeta^2}\ \bar{\sigma}^{\beta\,\zeta^2}
  & \rightarrow \qquad\qquad  & 
    \vepbd_{tr} = (F-1)\ \vepbd_{ss}
    \qquad , \qquad
    \vepbd_{p} = F\ \vepbd_{ss}
  \\[10pt]
    \varepsilon_{tr} = \varepsilon_{tr}^{n} + \Delta t\ \vepbd_{tr}
  &&
  \varepsilon_{p} = \varepsilon_{p}^{n} + \Delta t\ \vepbd_{p} \\[15pt]
\end{array}\]
Note that MD formulation uses $\log_{10}$ operator for the $\omega$ parameters.
I chose to remove it, because $\omega$ is a fitting parameter.
The idea is to keep the formulation clearer.
Note, however, that this changes the interpretation of the material fitting parameter $\beta$.
% \[\begin{array}{ c L }
% \hline
% E=mc^2 & any text\\ \hline
% cell4 & cell5 \\ \hline
% \end{array}\]

\subsection{Creep integration with large strain elasticity}
For small timesteps, we can determine the creep strain variation explicitely:
\begin{align*}
  \eps^s (t+\Delta t) = \eps^s(t) + \Delta \eps^s(t+\theta) 
\end{align*}
where $\theta$ indicates the time when the increment in creep strain is measured.
We start using an explicit measurement, $\theta=0$ (stable for small $\Delta t$), so that
\begin{align*}
\Delta \eps^s=\dot{\eps}(t)\ \Delta t
\end{align*}
Before moving to the next timestep, recalculate the stresses using the above creep strain.
Check for a threshold and reduce the time increment if necessary.

Rewrite the mechanical equilibrium equation with the creep increments.
We know how to solve elastic stresses, so:
\begin{align*}
  &  \cF = ( W_{i,I},\  P_{iI}^e )_{\Omega_0} - ( W_i,\ \tilde{T}_i )_{\Gamma_H} = 0\\
  &  P_{iI}^e = P_{iI} - P_{iI}^s 
\end{align*}
We need to move the creep contribution $P_{iI}^s$ to the current configuration:
\begin{align*}
 &  P_{iI}^s  = J \ \sigma_{ij}^s \ F^{-1}_{Ij} \\
 &  \cF  = (W_{i,I},\ P_{iI})_{\Omega_0} - ( W_{i,I},\  J \ \sigma_{ij}^s \ F^{-1}_{Ij} )_{\Omega_0} - ( W_i,\ \tilde{T}_i )_{\Gamma_H} = 0
\end{align*}
Now we must calculate $\sigma_{ij}^s$:
\begin{align*}
  &  \sigma_{ij}^s = \cc_{ijkl}\ \varepsilon_{kl}^s \\
  & \cc_{ijkl} = J^{-1}\ F_{iI}F_{jJ}F_{kK}F_{lL}\  \CC_{IJKL}
\end{align*}
\begin{align*}
  &  P_{iI}^e = F_{iJ}\ S_{JI} \\
  &  S_{JI}^e = \CC_{JIKL}\ E_{KL}^e \\
  &  E_{KL}^e = E_{KL} - E_{KL}^s
\end{align*}
where $E_{KL}^e$ is the elastic behavior of the Green-Lagrange strain and $E_{KL}^s$ is the GL strain for the secondary creep.  We now need to compute $E_{KL}^s$ from $\eps^s$:
\begin{align*}
  \varepsilon^e_{ij}
\end{align*}

\newpage
\section{Large Strains formulation for elasticity}

%
%
%
%
%
\subsection{Some definitions}

The deformation gradient $\F$:
\begin{align}
  \F = \frac{\partial \bm{x}}{\partial \bm{X}} = \frac{\partial\u}{\partial\bm{X}} + \bm{I} 
\end{align}
or, equivalently
\begin{align}
  F_{iJ} = u_{i,J} + \delta_{iJ}
\end{align}
and
\begin{align}
  J= \det{\F} \quad.
\end{align}
First and second Piola Kirchhoff
\begin{align}
& P_{iI} = F_{iJ}\ S_{JI} \\
& \CC_{IJKL} = \frac{\partial S_{IJ}}{\partial E_{KL}} \\
& \cc_{ijkl} = J^{-1}\ F_{iI}F_{jJ}F_{kK}F_{lL}\  \CC_{IJKL}
\end{align}



%
%
%
%
%
\subsection{Elasticity derivation}

Articulating things in the initial configuration:
\begin{align}
  & \nabla \cdot P = 0 & \text{on }\Omega_0 \\
  & U = \tilde{U}      & \text{on }\Gamma_G \\
  & T=P\ N=\tilde{T}      & \text{on }\Gamma_H 
\end{align}

The weak form is
\begin{align}
  \cF = ( W_{i,I},\  P_{iI} )_{\Omega_0} - ( W_i,\ \tilde{T}_i )_{\Gamma_H} = 0
\end{align}
We want to linearize $\cF$.
\begin{align}
  D\cF(\varphi)\cdot\Delta U_i =
       \frac{d}{d\varepsilon}\bigg( W_{i,I},\ P_{iI}(\varphi+\varepsilon\Delta U_i) \bigg)_{\Omega_0} \bigg|_{\varepsilon\rightarrow 0}
\end{align}
Let's work on $P$ and derive wrt $\varepsilon$.
\begin{align}
& P = F_{iJ}\ S_{JI} = \frac{\partial\varphi_i}{\partial X_J}\ S_{JI}(\varphi_i) \\
& P(\varphi_i + \varepsilon \Delta U_i) =  
                         \frac
                             {\partial (\varphi_i + \varepsilon \Delta U_i)}
                             {\partial X_J}\ S_{JI} (\varphi_i + \varepsilon \Delta U_i) \\
\end{align}
\begin{align}
\frac{\partial P (\varphi_i + \varepsilon \Delta U_i)}{\partial \varepsilon} 
& =
                         \frac {\partial\Delta U_i}{\partial X_J}\ S_{JI} +
                         \frac{\partial \varphi_i}{\partial X_J}\ \frac{\partial S_{JI}}{\partial \varepsilon}\\
&= 
                        \Delta U_{i,J}\  S_{JI} +
                        F_{iJ} \frac{\partial S_{JI}}{\partial E_{KL}}
                         \frac{\partial E_{KL} (\varphi_i + \varepsilon \Delta U_i) }{\partial \varepsilon} \\
& =
                        \Delta U_{i,J}\  S_{JI} +
                         F_{iJ}\ \CC_{JIKL}
                         \frac{\partial E_{KL} (\varphi_i + \varepsilon \Delta U_i) }{\partial \varepsilon} 
  \label{eqn:piola_linearization}
\end{align}
Recall that
\begin{align}
  2 E_{KL} & = F_{lK}\ F_{lL} - \delta_{KL} \\
           &= 
                \frac{\partial \varphi_l}{\partial X_K} 
                \frac{\partial \varphi_l}{\partial X_L} 
                - \delta_{KL}
\end{align}
and
\begin{align}
  2 \lim_{\varepsilon\to0} \partial_\varepsilon E_{KL}  (\varphi_l + \varepsilon \Delta U_l )
      &= \partial_\varepsilon F_{lK}\ F_{lL} + \partial_\varepsilon F_{lL}\ F_{lK}  \\
      &= \Delta U_{l,K}\ F_{lL} + \Delta U_{l,L}\ F_{lK}
\end{align}
Observing the $K$-$L$ simmetry in $C_{JIKL}$, we can say that
\begin{align}
   C_{JIKL}\ \lim_{\varepsilon\to0} \partial_\varepsilon E_{KL} (\varphi_l + \varepsilon \Delta U_l) 
        & = C_{JIKL}\ F_{lK}\ \Delta U_{l,L}
\end{align}
Subs back into the \eqref{eqn:piola_linearization} to obtain
\begin{align}
  \partial_\varepsilon P &=
  \Delta U_{i,J}\ S_{JI} + F_{iJ}\ F_{lK}\ \CC_{JIKL}\ \Delta U_{l,L}
\end{align}
The inner product becomes:
\begin{align}
  D \cF(\varphi_l) \cdot \Delta U_l = \left( W_{i,I} \quad,\quad U_{i,J}\ S_{JI} + F_{iJ}\ F_{lK}\ \CC_{JIKL}\ \Delta U_{l,L} \right)
\end{align}
And the linearized stiffness matrix:
\begin{align}
  K_{il}^{\beta\gamma} = \int_\Omega \bigg[
                               \phi^\beta_{,I}\ \phi^\gamma_{,J}\ S_{JI}\ \delta_{il} + 
                             \phi^\beta_{,I}\ \phi^\gamma_{,L}\ F_{iJ}\ F_{lK} \CC_{JIKL} 
                           \Bigg]\ d\Omega
\end{align}
%
%
%
%
%
\subsection{Pressure loading - calculations using $F$ (initial configuration).}

In the current configuration, write:
\begin{align}
  & h_i = -p\ n_i                             & \text{on}\ \Gamma_h \\
  & \int_{\Gamma_h} w_i\ h_i\ d\Gamma = 
  - \int_{\Gamma_h} p\ \w\cdot\n\ d\Gamma         & \text{on}\ \Gamma_h \\
\end{align}
where $p$ is the constant pressure (normal force) at the boundary. 
Note that the coordinate system is not constant, so the force direction changes with time.

Recall cramers rule:
\begin{align}
& \cof{\F^T} = \F^{-1} \det(\F) \\
& \cof{\F} = \F^{-T} \det(\F) = J\ \F^{-T} \\
& [\cof{\F}]_{iI} = J\ (F^{-1})_{Ii} 
\end{align}
We know that
\begin{align}
\n\ da = J \F^{-T}\ \N\ dA = \cof{(\F)}\ \N\ dA\\
n_i\ da = J [\F^{-1}]_{Ii}\ N_I\ dA = [\cof{(\F)}]_{iI}\ N_I\ dA
\end{align}
where
\begin{align}
  \cof{(\F)}_{iI} = (-1)^{i+I} M_{iI}
\end{align}
and $M_{iI}$ is the minor of the element $iI$.
We can now write
\begin{align}
   - \int_{\Gamma_h} p\ n_i\ w_i\ d\Gamma =
   & - \int_{\Gamma_H} p\ \W\cdot J F^{-T} \N \ d\Gamma = \\ 
   & - \int_{\Gamma_H} p\ \W\cdot \left[ \cof{(\F)}\ \N \right]\ d\Gamma  \\
   & - \int_{\Gamma_H} p\ W_i (J\ F^{-1})_{Ii}\ N_I \ d\Gamma = 
\end{align}
Now we need to linearize the above in the initial configuration.
Preliminary derivation (Jacobi formula):
\begin{align}
  \partial J = \partial \det \F 
    &= \frac{d [\det\F] }{d F_{jJ}} \ \partial F_{jJ} \\
    &= \det \F\ [\F^{-T}]_{jJ} \ \partial F_{jJ} \\
    &= J\ [\F^{-T}]_{jJ} \ \partial F_{jJ} \\
    &= J\ [\F^{-1}]_{Jj} \ \partial F_{jJ} \\
    &= \det \F \tr \left\{\F^{-1} \ \partial \F \right\} \\
    &= J \left[\F^{-1} \ \partial \F \right]_{kk} \\
    &= \left[ J\ \F^{-1} \ \partial \F \right]_{kk} \\
    &= \left[ \cof\F^T \ \partial \F \right]_{kk}  \\
    &= \left[ \adj\F \ \partial \F \right]_{kk} 
\end{align}
The derivative of an inverse matrix:
\begin{align}
  &\F\ \F^{-1} = \bm{I}\\
  \rightarrow\quad & \partial \left[\F \F^{-1}\right]=0 \\
  \rightarrow\quad & \F\ \partial \left[\F^{-1}\right] + \partial \F\ \F^{-1} = 0 \\
  \rightarrow\quad & \F\ \partial \left[\F^{-1}\right] = - \partial \F \ \F^{-1} \\
  \rightarrow\quad \partial \left[\F^{-1}\right] & = - \F^{-1}\ \partial\F\ \F^{-1} \\
  \rightarrow\quad \partial \left[\F^{-1}\right]_{Ii}
       &= - \left[\F^{-1}\right]_{Ij} \left[\partial \F\right]_{jJ} \left[\F^{-1}\right]_{Ji}
\end{align}
Now we can derive the derivative of the cofactors:
\begin{align}
  \partial \left[\cof \F \right] = \partial \left[\cof \F \right]_{iI} 
  &= \partial \left[ J \F^{-T} \right] \\
  &= \left[\F^{-T}\right] \partial J + J\ \partial\left[\F^{-T}\right] \\
  &= \left[\F^{-T}\right]_{iI} \partial J + J\ \partial\left[\F^{-T}\right]_{iI} \\
  &= \left[\F^{-1}\right]_{Ii} \partial J + J\ \partial\left[\F^{-1}\right]_{Ii} \\
  &= \left[\F^{-1}\right]_{Ii}\ J\ [\F^{-1}]_{Jj} \ \partial \left[\F\right]_{jJ}  -
    J \left[\F^{-1}\right]_{Ij} \left[\partial \F\right]_{jJ} \left[\F^{-1}\right]_{Ji} \\
  &= J \partial \left[\F\right]_{jJ}\Bigg\{
          \left[\F^{-1}\right]_{Ii}\ [\F^{-1}]_{Jj} \ -\ 
          \left[\F^{-1}\right]_{Ij}\  \left[\F^{-1}\right]_{Ji} 
      \Bigg\}
\end{align}
and the directional derivative becomes
\begin{align}
  \partial \left[ \cof \F(\varphi_i + \varepsilon \Delta u_i) \right]_{iI}
  &= \Bigg\{ 
           \left[\F^{-1}\right]_{Ii}\ [\F^{-1}]_{Jj} \   - 
           \left[\F^{-1}\right]_{Ij} \left[\F^{-1}\right]_{Ji} 
    \Bigg\}\  J\ \Delta u_{j,J} \\
\end{align}
And finally we transport ths restult to the linearization of the pressure loading.  Let
\begin{align}
   \cF(\varphi_i) &= - \int_{\Gamma_H} p\ \W\cdot \bigg[ \cof{\bigg(\F(\varphi)\bigg)}\ \N \bigg]\ d\Gamma  \\
                  &= - \int_{\Gamma_H} p\ W_i\ \left[\cof{(\F(\varphi)}\right]_{iI}\ N_I \ d\Gamma  \\
   \F &= \frac{d \x}{d \X} \\
   F_{iI} &= \frac{d x_i}{d X_I} = \frac{d\varphi_i}{d X_I}\\
   F_{iI}(\varphi_i + \varepsilon\Delta u_i) 
          &= \frac{d \left(\varphi_i + \varepsilon\Delta u_i\right)}{d X_I}\\
   \lim_{\varepsilon\to 0} \partial_\varepsilon F_{jJ}(\varphi_k + \varepsilon\Delta u_k) 
          &= \frac{d \Delta u_k}{d X_K} = \Delta u_{j,J}
\end{align}
Then
\begin{align}
  \lim_{\varepsilon\to 0} \partial_\varepsilon \cF(\varphi+\varepsilon\Delta\u) 
   &= - \int_{\Gamma_H} p\ \W\cdot 
          \bigg\{ \partial_\varepsilon \cof{\bigg(\F(\varphi+\varepsilon\Delta\u)\bigg)}\ 
           \N \bigg\}\ d\Gamma  \\
   &= - \int_{\Gamma_H} p\ J\ W_i\ N_I\ 
          \bigg\{ 
                 \left[\F^{-1}\right]_{Ii} [\F^{-1}]_{Jj} -
                 \left[\F^{-1}\right]_{Ij} \left[\F^{-1}\right]_{Ji} 
           \bigg\}\ \Delta u_{j,J}\ d\Gamma   \\
   \\
   &= - \int_{\Gamma_H} p\ J\ W_i\ N_I\ 
          \bigg\{ 
            \left[\F^{-1}\right]_{Ik} [\F^{-1}]_{Jl} \delta_{ik}\delta_{jl} -
            \left[\F^{-1}\right]_{Ik} \left[\F^{-1}\right]_{Jl} \delta_{il} \delta_{jk}
           \bigg\}\ \Delta u_{j,J}\ d\Gamma   \\
   &= - \int_{\Gamma_H} p\ J\ W_i\ N_I\ \left[\F^{-1}\right]_{Ik} [\F^{-1}]_{Jl}\ 
   \big( \delta_{ik}\delta_{jl} - \delta_{il} \delta_{jk} \big)\ 
          \Delta u_{j,J}\ d\Gamma   \\
   &= \int_{\Gamma_H} p\ J\ W_i\ N_I\ \left[\F^{-1}\right]_{Ik} [\F^{-1}]_{Jl}\ 
   \big( \delta_{il} \delta_{jk} - \delta_{ik}\delta_{jl}  \big)\ 
          \Delta u_{j,J}\ d\Gamma   \\
\end{align}
The element matrix becomes:
\begin{align}
  K_{ij}^{\beta\gamma}
   &= \int_{\Gamma_H} \phi^\beta \phi_{,J}^\gamma\ p\ J\ N_I\ \left[\F^{-1}\right]_{Ik} [\F^{-1}]_{Jl}\ 
   \big( \delta_{il} \delta_{jk} - \delta_{ik}\delta_{jl}  \big)\ d\Gamma
\end{align}
%
%
%
%
%
\subsection{Pressure loading - calculating from the isoparametric state}

Recall that 
\begin{align}
  \def\w{\bm{w}}
  \def\xh{\hat{\bm{x}}}
  \def\cross{\left(\frac{\partial\xh}{\partial\xi_1} \times \frac{\partial\xh}{\partial\xi_2}\right)}
  d\Gamma = \left| \cross \right| d\Gamma^\square
\end{align}
where $\Gamma^\square$ is the surface in the reference (isoparametric) configuration.
We can calculate the normal vector in the current configuration using
\begin{align}
  \def\w{\bm{w}}
  \def\xh{\hat{\bm{x}}}
  \def\cross{\left(\frac{\partial\xh}{\partial\xi_1} \times \frac{\partial\xh}{\partial\xi_2}\right)}
  n_i=\frac{\cross}{\left|\cross\right|}
\end{align}
so that
\begin{align*}
  \def\w{\bm{w}}
  \def\xh{\hat{\bm{x}}}
  \def\cross{\left(\frac{\partial\xh}{\partial\xi_1} \times \frac{\partial\xh}{\partial\xi_2}\right)}
  \int_{\Gamma_h} w_i\ h_i\ d\Gamma = -\int_{\Gamma_h^\square} p\ \w \cdot \cross d\Gamma^\square
%   & \int_{\Gamma_h} w_i\ h_i\ d\Gamma = - \int_{\Gamma_h} p\ w_i\ \cross d\Gamma         & \text{on}\ \Gamma_h
\end{align*}
Now we need to compute $\frac{\partial\hat{x}}{\partial\xi_i}$. So
\begin{align*}
  \def\w{\bm{w}}
  \def\xh{\hat{\bm{x}}}
  \def\Xh{\hat{\bm{X}}}
\frac{\partial\xh}{\partial\xi_i} = \frac{\partial \phi^\gamma}{\partial\xi_i} U^\gamma = \F\ \frac{\partial\Xh}{\partial\xi_i}  \\
\def\Xh{\hat{\bm{X}}}
\frac{\partial\Xh}{\partial\xi_i} = \frac{\partial \phi^\gamma}{\partial\xi_i}\ X_i^\gamma \\
\end{align*}
We finally obtain

\def\w{\bm{w}}
\def\xh{\hat{\bm{x}}}
\def\Xh{\hat{\bm{X}}}
\def\crossFX{\left(\F \frac{\partial\Xh}{\partial\xi_1} \times \F \frac{\partial\Xh}{\partial\xi_2}\right)}
\def\crossFXi{\left(\F \frac{\partial\Xh}{\partial\xi_1} \times \F \frac{\partial\Xh}{\partial\xi_2}\right)_i}
\def\cross{\left(\frac{\partial\xh}{\partial\xi_1} \times \frac{\partial\xh}{\partial\xi_2}\right)}
\begin{align*}
  \cF = \int_{\Gamma_h} w_i\ h_i\ d\Gamma &= -\int_{\Gamma_h^\square} p\ \W \cdot \cross d\Gamma^\square \\
                                          &= -\int_{\Gamma_h^\square} p\ \W \cdot \left( \frac{\partial \U}{\partial\xi_1}\times\frac{\partial \U}{\partial\xi_2}\right) d\Gamma^\square \\
                                          &= -\int_{\Gamma_h^\square} p\ \W \cdot \left( \frac{\partial \phi^\gamma}{\partial\xi_1}\U^\gamma\times\frac{\partial \phi^\gamma }{\partial\xi_2}\U^\gamma\right) d\Gamma^\square \\
\end{align*}
Now we can linearize (see Belytschko book, pg365)
\begin{align*}
  D\cF\cdot\Delta U &= \lim_{\varepsilon\to0}\partial_\varepsilon\cF(\varphi+\varepsilon\Delta U) \\
                    &=\lim_{\varepsilon\to0} \Bigg[ -\int_{\Gamma_h^\square} p\ W_i \ \ \partial_\varepsilon \crossFXi d\Gamma^\square \Bigg]\\
                    &=\lim_{\varepsilon\to0} \Bigg[ -\int_{\Gamma_h^\square} p\ W_i 
                      \Bigg(
                        \partial_\varepsilon \hat{\x}_{,\xi_1} \times \hat{\x}_{,\xi_2} + 
                        \hat{\x}_{,\xi_1} \times \partial_\varepsilon \hat{\x}_{,\xi_2}  
                      \Bigg)_i\\
                    &= -\int_{\Gamma_h^\square} p\ W_i 
                      \Bigg(
                        \Delta \U_{,\xi_1} \times \hat{\x}_{,\xi_2} + 
                        \hat{\x}_{,\xi_1} \times \Delta \U_{,\xi_2}  
                      \Bigg)_i\\
                    &= -\int_{\Gamma_h^\square} p\ W_i 
                      \Bigg(
                        \Delta \U_{,\xi_1} \times \F \frac{\partial\Xh}{\partial\xi_2} + 
                        \F \frac{\partial\Xh}{\partial\xi_1} \times \Delta \U_{,\xi_2}  
                      \Bigg)_i\\
                    &= -\int_{\Gamma_h^\square} p\ W_i 
                      \Bigg(
                        \Delta U_{j,\xi_1} \times F_{jJ} \frac{\partial\hat{X}_J}{\partial\xi_2} + 
                        F_{kK} \frac{\partial\hat{X}_K}{\partial\xi_1} \times \Delta U_{k,\xi_2}  
                      \Bigg)_i\\
  \\D\cF\cdot\Delta U &= \lim_{\varepsilon\to0}\partial_\varepsilon\cF(\varphi+\varepsilon\Delta U) \\
                    &=\lim_{\varepsilon\to0} \Bigg[ -\int_{\Gamma_h^\square} p\ W_i \ 
                      \Bigg(
                        \partial_\varepsilon \F \frac{\partial\Xh}{\partial\xi_1} \times \F \frac{\partial\Xh}{\partial\xi_2}
                       \ +\ 
                        \F \frac{\partial\Xh}{\partial\xi_1} \times \partial_\varepsilon \F \frac{\partial\Xh}{\partial\xi_2}
                      \Bigg)
                    d\Gamma^\square \Bigg]\\
                    &=-\int_{\Gamma_h^\square} p\ W_i \ 
                      \Bigg(
                        \nabla (\Delta \bm{U}) \frac{\partial\Xh}{\partial\xi_1} \times \F \frac{\partial\Xh}{\partial\xi_2}
                       \ +\ 
                       \F \frac{\partial\Xh}{\partial\xi_1} \times \nabla (\Delta \bm{U}) \frac{\partial\Xh}{\partial\xi_2}
                      \Bigg)_i
                    d\Gamma^\square  \\
                    &=-\int_{\Gamma_h^\square} p\ W_i \ 
                      \Bigg(
                        \Delta U_{j,I} \frac{\partial\hat{X}_I}{\partial\xi_1} F_{kJ} \frac{\partial\hat{X}_J}{\partial\xi_2} 
                       \ +\ 
                       F_{jI} \frac{\partial\hat{X}_I}{\partial\xi_1} \Delta U_{k,J} \frac{\partial\hat{X}_J}{\partial\xi_2} 
                      \Bigg)\ \varepsilon_{jki}
                    \ \ d\Gamma^\square  \\
                    &=-\int_{\Gamma_h^\square} p\ W_k \ 
                      \Bigg(
                        \Delta U_{i,I} \frac{\partial\hat{X}_I}{\partial\xi_1} F_{jJ} \frac{\partial\hat{X}_J}{\partial\xi_2} 
                       \ +\ 
                       F_{iI} \frac{\partial\hat{X}_I}{\partial\xi_1} \Delta U_{j,J} \frac{\partial\hat{X}_J}{\partial\xi_2} 
                      \Bigg)\ \varepsilon_{ijk}
                    \ \ d\Gamma^\square  \\
                    &=-\int_{\Gamma_h^\square} p\ W_k \ 
                      \Big( F_{jJ}\ \Delta U_{i,I} \ +\ F_{iI}\ \Delta U_{j,J} \Big)\ 
                      \frac{\partial\hat{X}_I}{\partial\xi_1}\ \frac{\partial\hat{X}_J}{\partial\xi_2} 
                      \ \varepsilon_{ijk}
                    \ \ d\Gamma^\square  \\
                    &=-\int_{\Gamma_h^\square} p\ W_k \ \Delta U_{i,I} \ F_{jJ}\ 
                      \frac{\partial\hat{X}_I}{\partial\xi_1}\ \frac{\partial\hat{X}_J}{\partial\xi_2} 
                      \ \varepsilon_{ijk}
                    \ \ d\Gamma^\square  -
                    \int_{\Gamma_h^\square} p\ W_k \ \Delta U_{j,J} \ F_{iI}\  
                      \frac{\partial\hat{X}_I}{\partial\xi_1}\ \frac{\partial\hat{X}_J}{\partial\xi_2} 
                      \ \varepsilon_{ijk}
                    \ \ d\Gamma^\square  \\
                    &= -\  W_k^\beta \ \Delta U_i^\gamma \int_{\Gamma_h^\square} p\ \phi^\beta  \ \phi^\gamma_{,I} \ F_{jJ}\ 
                      \frac{\partial\hat{X}_I}{\partial\xi_1}\ \frac{\partial\hat{X}_J}{\partial\xi_2} 
                      \ \varepsilon_{ijk} \ d\Gamma^\square  \\ 
                    &\quad -
                    \ W_k^\beta \ \Delta U_j^\gamma \int_{\Gamma_h^\square} p\ \phi^\beta \ \phi^\gamma_{,J} \ F_{iI}\  
                      \frac{\partial\hat{X}_I}{\partial\xi_1}\ \frac{\partial\hat{X}_J}{\partial\xi_2} 
                      \ \varepsilon_{ijk} \ d\Gamma^\square  \\
\end{align*}
where
\begin{align}
  \varepsilon_{ijk} =
  \begin{cases}
    +1 \quad \text{if } (ijk)=(123),(231),(312) - \text{even permutation}\\
    -1 \quad \text{if } (ijk)=(321),(132),(213) - \text{odd permutation}\\
    0  otherwise
  \end{cases}
\end{align}
\newpage
\section{Incremental formulation}
Start from the residual weak formulation.
\begin{align*}
  \cF &= ( W_{i,I},\  P_{iI} )_{\Omega_0} - ( W_i,\ \tilde{T} )_{\Gamma_H} = 0 \\
      &= ( W_{i,I},\  F_{iJ}\ S_{JI} )_{\Omega_0} - ( W_i,\ \tilde{T} )_{\Gamma_H} = 0 \\
      &= ( W_{i,I},\  F_{iJ}\ \CC_{JIKL}\ E_{KL} )_{\Omega_0} - ( W_i,\ \tilde{T} )_{\Gamma_H} = 0 
\end{align*}
We need to define the timestep of the variables.
Define the states $0$, $1$, $2$ as the initial, last solved and next to solve respectively.
Then
\begin{align*}
  E_{KL} &= \prescript{2}{0}{E}_{KL} = \prescript{1}{0}{E}_{KL} + \prescript{1\rightarrow2}{0}{\delta E}_{KL} \\
  \delta E_{KL} &= \prescript{1\rightarrow2}{0}{\delta E}_{KL} = \prescript{2}{0}{E}_{KL} - \prescript{1}{0}{E}_{KL} \\
\nonumber\\
  S_{JI} &= \prescript{2}{0}{S}_{JI} \\
         & = \prescript{1}{0}{S}_{JI} + \prescript{2}{0}{\delta S}_{JI} \\
\end{align*}
Calculate $E_{KL}$ from the displacements:
\begin{align*}
  2 \delta E_{KL} &=
  \prescript{2}{0}{\Bigg[U_{K,L} + U_{L,K} + U_{i,K}\ U_{i,L}\Bigg]} -
  \prescript{1}{0}{\Bigg[U_{K,L} + U_{L,K} + U_{i,K}\ U_{i,L}\Bigg]}  \\
                  &= \Bigg( \delta U_{K,L} + \delta U_{L,K} +
                  \prescript{1}{0} U_{i,K}\ \delta U_{i,L} +
                  \prescript{1}{0} U_{i,L}\ \delta U_{i,K} \Bigg) +
                  \Bigg( \delta U_{i,L} \delta U_{i,K} \Bigg)
\end{align*}
Back to the formulation:
\begin{align*}
  \cF &=
          \left( W_{i,I},\quad F_{iJ}\ C_{JIKL}\ \prescript{1}{0}{E_{KL}} \right) +
          \left( W_{i,I},\quad F_{iJ}\ C_{JIKL}\ \delta E_{KL} \right) -
          (W_i, \tilde{T} ) \\
      &= \left( W_{i,I},\quad F_{iJ}\ C_{JIKL}\ \prescript{1}{0}{E_{KL}} \right)  \\
      & + \left( W_{i,I},\quad F_{iJ}\ C_{JIKL}\ \delta U_{K,L} \right) \\
      & + \left( W_{i,I},\quad F_{iJ}\ C_{JIKL}\ \delta U_{L,K} \right) \\
      & + \left( W_{i,I},\quad F_{iJ}\ C_{JIKL}\ \prescript{1}{0} U_{i,K}\ \delta U_{i,L} \right) \\
      & + \left( W_{i,I},\quad F_{iJ}\ C_{JIKL}\ \prescript{1}{0} U_{i,L}\ \delta U_{i,K} \right) \\
      & + \left( W_{i,I},\quad F_{iJ}\ C_{JIKL}\ \delta U_{i,L}\ \delta U_{i,K} \right) \\
      & - (W_i,\quad \tilde{T} )_{\Gamma_H}
\end{align*}
Linearize:
\begin{align*}
  D\cF\cdot\delta U_i & = \lim_{\varepsilon\to0} \partial_\varepsilon \cF(\delta U +\varepsilon \Delta U_i) \\
      & = \left( W_{i,I},\quad F_{iJ}\ C_{JIKL}\ \Delta U_{K,L} \right) \\
      & + \left( W_{i,I},\quad F_{iJ}\ C_{JIKL}\ \Delta U_{L,K} \right) \\
      & + \left( W_{i,I},\quad F_{iJ}\ C_{JIKL}\ \prescript{1}{0} U_{i,K}\ \Delta U_{i,L} \right) \\
      & + \left( W_{i,I},\quad F_{iJ}\ C_{JIKL}\ \prescript{1}{0} U_{i,L}\ \Delta U_{i,K} \right) \\
      & + \left( W_{i,I},\quad F_{iJ}\ C_{JIKL}\ \delta U_{i,L}\ \Delta U_{i,K} \right)  \\
      & + \left( W_{i,I},\quad F_{iJ}\ C_{JIKL}\ \Delta U_{i,L}\ \delta U_{i,K} \right) 
\end{align*}
Trying from the deformation gradient perspective:
\begin{align*}
  \delta E_{KL} &= \big( \prescript{2}{0}{\F^T}\ \prescript{2}{0}{\F} - \I \big) -\ \big( \prescript{1}{0}{\F^T}\ \prescript{1}{0}{\F} - \I \big) \\
                &=  \prescript{2}{0}{\F^T}\ \prescript{2}{0}{\F} -\ \prescript{1}{0}{\F^T}\ \prescript{1}{0}{\F}  \\
                &=  \prescript{2}{0}{F_{iK}}\ \prescript{2}{0}{F_{iL}} -\ \prescript{1}{0}{F_{iK}}\ \prescript{1}{0}{F_{iL}} 
\end{align*}
Hence:
\begin{align*}
  \cF &=
          \left( W_{i,I},\quad F_{iJ}\ C_{JIKL}\ \prescript{1}{0}{E_{KL}} \right) +
          \left( W_{i,I},\quad F_{iJ}\ C_{JIKL}\ \delta E_{KL} \right) -
          (W_i, \tilde{T} ) \\
      &= 
          \left( W_{i,I},\quad F_{iJ}\ C_{JIKL}\ \prescript{1}{0}{E_{KL}} \right) \\
      &+  \left( W_{i,I},\quad F_{iJ}\ C_{JIKL}\ \prescript{2}{0}{F_{iK}}\ \prescript{2}{0}{F_{iL}} \right) \\
      &- \left( W_{i,I},\quad F_{iJ}\ C_{JIKL}\ \prescript{1}{0}{F_{iK}}\ \prescript{1}{0}{F_{iL}}\right) \\
      &-    (W_i, \tilde{T} ) \\
\end{align*}
Did not work out, because the $\Delta U_i$ does not show. Need to open the terms, and will get to the same verbosity.
\newpage
\printbibliography
\end{document}
